
\title{DIKU Sommerkursus: Programmering Brush-Up}
\author{}
\documentclass[11pt,a4paper]{article}
\usepackage[utf8]{inputenc}

\begin{document}
\maketitle


\section{Formål}
Formålet med sommerkurset er at give nye bachelorstuderende med lille eller ingen
programmeringserfaring en mulighed for at opnå praktisk programmeringserfaring   
inden studiestart. Forhåbentligt vil dette medføre at dette vil formindske den
tydelige forskel i forhold til de mere erfarne studerende. % Frafald?? 

\section{Kursusbeskrivelse}
Kurset vil være et 5-dages turbokursus der introducerer de studerende til simple
imperativ programmering. 
% lala
Kurset vil være stuktureret meget som den undervisningsform de allerede kender
fra gymnasiet. Dvs. at alle studerende inddeles i klasser af 15-20 personer der
hver har en tilknyttet en fast instruktior. Alt undervisning foregår så i disse
klasser. En undervisningsdag vil vare fra 09:00 - 16:00, inklusiv frokkostpause,
hvilket giver 6 timers undervisning.
\\\\
Hver dag vil forløbe således: Om morgnen mødes de studerende i hvert deres 
klasselokale med deres instruktor der vil undervise i dagens tema. Her efter vil 
de studerende løse en række små træningsopgaver med hjælp fra instruktoren så de
med det samme opnår en praktisk erfaring. Sidst på dagen vil de så løse dagens
hovedopgave (igen med hjælp fra instruktor). Hovedopgaven er designet til at 
bruge dagens tema i en større kontekst. Opgaven vil have en udformning der 
muliggør at de studerende mere frit kan eksperimentere med den, samt have en 
forudprogrammeret GUI så de får indtryk af at have produceret et reelt stykke 
software.

\section{Kursusforløb}
\begin{description}
  \item[Mandag] Introduktion: værdier, variable og simple funktioner.
  \item[Tirsdag] Kontrolflow: if, for while.
  \item[Onsdag] Avancerede funtioner og kontrolflow (opsamling).
  \item[Torsdag] Klasser og objekter (simpelt!).
  \item[Fredag] Miniprojekt: n-body simulator.
\end{description}


\section{Teknik}
Vi har tænkt os at undervisningen skal foregå i programmeringssproget 
CoffeeScript\footnote{http://coffeescript.org/}. Fordelene ved CoffeeScript
til denne form for begynderkursus er umidlbare: Sproget har en simpel og let 
tilgængelig syntax der læser sig mod Ruby og Python. Der ud over er sproget
browserbaseret og oversættes direkte til JavaScript. Dette muliggøre at alt 
software til brug i kurset vil være webbaseret og samlet på en enkelt hjemmeside.
De studerende skal altså ikke bruge tid på at downloade og installere en masse
software, de skal blot besøge en hjemmeside. Der vil til formålet blive udviklet
en side der fungerer som både editor, terminal og GUI.

%The idea of this sommer kurs is to intruduce the new studions ad diku to the basig conseps of programing. The course will mostly fucsus on studiens that have little or non programming experions and will primaly be about getting tools and praticle knowels in programing. The course will be a 5 dayes turbo course, just like the phisiks and matematick brusop courses and will ly in august, just before the start of the university. 
%We have two reasons that we think this course is importens. Firstly, there are a great divide between the level of programing knowes of new studions ad diku. This divide maks teaching diffecle and gaps appers in studion education, especially for thought wich have non or little programing experions. Secontly 


\end{document}
